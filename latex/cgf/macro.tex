% macros, d�finitions et nouvelles commandes perso
\def\argmax{\operatornamewithlimits{arg\,max}}
\def\argmin{\operatornamewithlimits{arg\,min}}
\newcommand{\dtc}{\ensuremath{\operatorname{d_{tc}}}}

\DeclareMathOperator*{\Mult}{Mult}


%\def\I{{\mathbf{I}}}
%\def\q{{\mathbf{q}}}
%\def\t{{\mathbf{t}}}
%\def\p{{\mathbf{p}}}
%\def\R{{\mathbb{R}}}
%\def\N{{\mathbb{N}}}
%\def\Z{{\mathbb{Z}}}
%\def\P{{\mathbb{P}}}
%\def\E{{\mathbb{E}}}
%\def\DD{{\mathcal{D}}}
%\def\HH{{\mathcal{H}}}
%\def\NN{{\mathcal{N}}}
%\def\JJ{{\mathcal{J}}}
%\def\eps{{\epsilon}}


\def\figurePath{fig/}
\def\myfigure#1#2#3{\begin{figure}[ht]\centering\includegraphics*[width = \linewidth]{\figurePath#2}\caption{#3}\label{fig:#1}\end{figure}}
\def\mycfigure#1#2#3{\begin{figure*}[t]\centering\includegraphics*[clip, width = \linewidth]{\figurePath#2}\caption{#3}\label{fig:#1}\end{figure*}}
\def\myfigurew#1#2#3#4{\begin{figure*}[t]\centering\includegraphics*[clip, width =#4]{\figurePath#2}\caption{#3}\label{fig:#1}\end{figure*}}

% For i.e. and e.g.
\newcommand{\ie}{i.\,e.\ }	% please leave it like this. either i.e AND e.g. need a comma afterwards or none
\newcommand{\eg}{e.\,g.\ }	% it's more a british (but not purely) thing to put a comma afterwards. but we use american style most of the time. like color samples
\newcommand{\etal}{et~al.\ }
\newcommand{\shortcite}[1]{\cite{#1}}
\usepackage[T1]{fontenc}
\newcommand{\bb}[1]{{\textbf #1}}
\newcommand{\R}{\mathrm{I\!R}}
\def \P {\mathcal{P}}

% Wraps sections with label
\def\mypart#1#2{\part{#1}\label{part:#2}}
\def\mychapter#1#2{\chapter{#1}\label{chapter:#2}}
\def\mysection#1#2{\section{#1}\label{sec:#2}}
\def\mysectionNoTOC#1#2{\section*{#1}\label{sec:#2}}
\def\mysubsection#1#2{\subsection{#1}\label{subsec:#2}}
\def\mysubsubsection#1#2{\subsubsection{#1}\label{subsubsec:#2}}
\def\myparagraph#1#2{\paragraph{#1}\label{paragraph:#2}}


\newcommand{\todo}[1]{[\textcolor{red}{\textbf{TODO:}} #1]}
\newcommand{\note}[1]{\textcolor{blue}#1}



\usepackage[normalem]{ulem}
\newcommand{\jtodel}[1]{\textcolor{red}{\sout{#1}}}
\newcommand{\jtoadd}[1]{\textcolor{blue}{#1}}
\renewcommand{\jtodel}[1]{}
\renewcommand{\jtoadd}[1]{#1}

\newcommand{\hiddenData}[1]{\textcolor{green}{\sout{#1}}}
\renewcommand{\hiddenData}[1]{}