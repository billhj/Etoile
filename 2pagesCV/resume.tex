\section{Current Position}
\cventry{2013--now}{PostDoc Researcher}{CNRS}{Paris}{}
{
\begin{itemize}
	\item work for "VIB" system
	\begin{enumerate}
	\item kinematics and dynamics system, physics simulation on multi-body objects, robotics (featherstone, lagrange rigid body dynamics lib in Java)
	\item emotion driven animation synthesis such as laughter (combine machine learning with rule based method)
	\item motion system recording for human characters, motion analysis
	\end{enumerate}
	%\item Uncanny Valley of rendering and animation for virtual agent simulations
\end{itemize}
}


\section{Education}
\cventry{2009--2013}{Doctorate degree}{Ecole Mine-Telecom de Paris}{Paris}{Computer Graphics}
{under direction of Catherine Pelachaud and Tamy Boubekeur (PhD thesis: 3D Emotional Rendering and Animation Models)
%\begin{itemize}
	%\item work for Real-time three dimensional embodied conversational agent system "Greta"
	%\item rendering and animation system design and building (using both C++ and JAVA)
	%\item Etoile-Edison system (engine using C++ and GPU programming)
%\end{itemize}
}  % arguments 3 to 6 can be left empty

\cventry{2007--2009}{Master Degree}{University of Paris Descartes (Paris 5) and Pasteur Institute}{Paris}{Computer Science and Bio-Technology, \textbf{top 5/100}}
{under direction of Florence Cloppet and Vannary Meas-Yedid (Master thesis: Segmentation Evaluation for microscopy images)
}
\cventry{2002--2006}{Bachelor Degree}{HangZhou Normal University}{HangZhou}{Bio-Technology}{}

\section{Work Experience}

\subsection{Related Projects}
\cventry{2015--2016}{Parallel Constrained Inverse Kinematics}{CNRS/Telecom ParisTech}{Paris}{}
{
a parallel high performance algorithm of Inverse Kinematics: learning IK parameters from real data by inversing process and fill constraint parameters octree, real-time synthesis computation can be made in parallel process by fast tree searching}

\cventry{2014--2015}{HMM based laughter synthesis}{CNRS/Telecom ParisTech}{Paris}{}
{
an algorithm based on hidden markov chain to find out the correlation between laughter phoneme and skeleton rotations for animation synthesis 
}

\cventry{2013--2014}{FPBGI}{Telecom ParisTech}{Paris}{}
{
a point based global illumination algorithm using a regional estimated starting travesaling node for octree based on the \textbf{Screen Space Clustering} by K-Means, significantly minimizing the travesaling time to improve the performance of PBGI
}

\cventry{2010--2013}{Virtual Character System}{Telecom ParisTech}{Paris}{}
{
\begin{itemize}
\item Animation behavior generating virtual character's interactions
\item Low level rendering using Ogre3D engine (c++, java and OpenGL/GLSL)
\end{itemize}
}

\cventry{2009--2012}{Etoile et Edison (Part-time Project)}{Telecom ParisTech}{Paris}{}
{
A plugin system and libraries (Windows/Linux): Rendering, Animation, DataMing, etc.
using C++, OpenGL/GLSL, CUDA
}

\cventry{2010--2011}{Separable Ambient Occlusion}{Telecom ParisTech}{Paris}{}
{
1st introducing separable computation in lighting simulation to reduce the complexity from $n^2$ to n
}

\subsection{Teaching Experience}
\cventry{2014--2016}{Course}{Unversity of Paris 6}{Paris}{}
{
C/C++ Programming ; JAVA Programming
}
\cventry{2014--2016}{Project}{Telecom ParisTech}{Paris}{}
{
supervisor of software engineering and JAVA programming; supervisor of projet PACT: an education system for hand writing recognition and text-grammar correction.
}
\cventry{2013--2015}{Course}{Telecom ParisSud}{Paris}{}
{
Modeling and encoding in 3D animation; Expressive Animation
}

%\section{Languages}
%\cvitemwithcomment{Chinese}{mother tongue}{}
%\cvitemwithcomment{English}{fluent}{}
%\cvitemwithcomment{French}{fluent}{}

\section{Computer skills}
\cvitem{Programming}{\textbf{C/C++}, \textbf{OpenGL/GLSL}, WebGL, HLSL, CUDA, OpenCL, \textbf{JAVA/J2EE}, JSF, Python, C$\sharp$, Basic, UML, 
XML, design patterns, \textbf{algorithm}, \textbf{Qt, Boost, OpenMP, tbb}}
%\cvitem{Data Base}{SQL, PL/SQL, Oracle (administration, Tuning, development, under 9i, 10g, 11g), PostgreSQL, MySQL (version 5.0), etc}
%\cvitem{Web}{HTML, PHP, CSS, JAVASCRIPT, etc}
\cvitem{Data Ming}{Weka, \textbf{Matlab}, Numpy, etc}
\cvitem{Learning}{Linear/Logistic regression, Neural Network, SVM, K-Means, HMM, ConvNets, etc}
\cvitem{Modeling}{\textbf{Blender}, 3DS Max, Unity3D, Photoshop, Illustrator, ImageJ, Adobe Premiere etc}
%\cvitem{Presentation}{\textbf{\LaTeX}, OpenOffice, Word, PowerPoint, etc}
\cvitem{System}{Linux, Embedded System(Galileo/Edison) etc}

%\section{Award}
%\cvlistitem{Prize "EGIDE" for foreign students in France, top 5 in master}



%\renewcommand{\listitemsymbol}{-~}            % change the symbol for lists

% Publications from a BibTeX file without multibib
%  for numerical labels: \renewcommand{\bibliographyitemlabel}{\@biblabel{\arabic{enumiv}}}
%  to redefine the heading string ("Publications"): \renewcommand{\refname}{Articles}
\nocite{*}
\bibliographystyle{plain}
\bibliography{publicationsOrig}                   % 'publications' is the name of a BibTeX file

\section{Review}
\cvlistitem{Computer Graphics Forum 2011}
\cvlistitem{Computer Animation and Virtual World 2014}
%\section{References}
%\cvlistitem{Dr. Catherine Pelachaud  \url{http://perso.telecom-paristech.fr/~pelachau/}}
%\cvlistitem{Dr. Tamy Boubekeur  \url{http://perso.telecom-paristech.fr/~boubek/}}
%\cvlistitem{Dr. Elmar Eisemann  \url{http://graphics.tudelft.nl/~eisemann/}}
%
%\section{Demo}
%\cvlistitem{website \url{http://perso.telecom-paristech.fr/~jhuang/}}
%\cvlistitem{youtube  \url{https://www.youtube.com/user/mrluckyhj/}}
