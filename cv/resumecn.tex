\section{姓名: 黄晶}
\section{当前职位}
\cventry{2013--now}{博士后研究员}{CNRS法國國家科學研究中心}{巴黎}{}
{
\begin{itemize}
	\item "Greta" "VIB" 人机互动系统
	\item 笑声情感驱动的动画合成系统 (欧洲项目 ilhaire)
	\item Uncanny Valley 虚拟动画及渲染
\end{itemize}
}


\section{教育背景}
\cventry{2009--2013}{博士学位}{Ecole Mine-Telecom de Paris 巴黎高科}{巴黎}{计算机图形学}
{博士导师: Catherine Pelachaud 和 Tamy Boubekeur (PhD 论文: 3D Emotional Rendering and Animation Models)
\begin{itemize}
	\item 三维人机互动系统 "Greta"
	\item 动画渲染系统 (C++ and JAVA)
	\item Etoile-Edison 系统 (三维引擎 C++ and GPU)
\end{itemize}
}  % arguments 3 to 6 can be left empty

\cventry{2007--2009}{硕士学位}{University of Paris Descartes (Paris 5)巴黎5大笛卡尔大学 and Pasteur Institute 巴斯德研究所}{巴黎}{计算机科学和生物技术}
{导师: Florence Cloppet 和 Vannary Meas-Yedid (Master 论文: Segmentation Evaluation for microscopy images)
}
\cventry{2002--2006}{本科学位}{杭州师范大学}{杭州}{生命科学}{}


\section{工作经历}

\subsection{GPU相关项目}
\cventry{2013--2013}{PBGI}{巴黎高科 CNRS法國國家科學研究中心}{巴黎}{}
{
GPU项目使用CUDA: Factorized Point-Based Global Illumination  基于点计算的全局照明
\begin{itemize}
	\item caching step: build point (sphere) based hierarchical structure and shade pixels with only the direct lighting and shadow
	\item rasterization step: rasterize spheres into each pixel's microbuffer with Z-buffer algorithm for visibility and convolve the BRDF for the pixel (the spheres projection process can be parallel using CUDA)
\end{itemize}
}

\cventry{2010--2013}{Virtual Character System 虚拟人物系统}{巴黎高科}{巴黎}{}
{
\begin{itemize}
\item High level emotion driven animation planner deciding human behavior
\item Animation behavior generating virtual character's interactions
\item Low level rendering using Ogre3D engine (c++, java and OpenGL/GLSL)
\end{itemize}
}

\cventry{2009--2012}{Etoile et Edison 系统}{巴黎高科}{巴黎}{}
{
A plugin system and libraries (Windows/Linux): Rendering, Animation, DataMing, etc.
using C++, OpenGL/GLSL, CUDA
}

\cventry{2010--2011}{Separable Ambient Occlusion 可分环境光照}{巴黎高科}{巴黎}{}
{
Propose a low computational cost, global illumination simulation approach for 3D rendering using OpenGL/GLSL.
\begin{itemize}
\item Noise Pattern Generations
\item Exploring Separable Technique Fashion
\item Using local randomizations with interleaved filtering techniques
\end{itemize}
}

\subsection{教学经验}
\cventry{2013--2014}{课程}{Telecom ParisSud}{巴黎}{}
{
动画渲染
}

\section{语言}
\cvitemwithcomment{中文}{母语}{}
\cvitemwithcomment{英语}{流利}{}
\cvitemwithcomment{法语}{流利}{}

\section{计算机技术}
\cvitem{Programming}{C/C++, OpenGL, WebGL, GLSL, HLSL, CUDA, OpenCL, JAVA/J2EE, JSF, Python, Basic, UML, 
XML, design patterns, etc}
\cvitem{Data Base}{SQL, PL/SQL, Oracle (administration, Tuning, development, under 9i, 10g, 11g), PostgreSQL, MySQL (version 5.0), etc}
\cvitem{Web}{HTML, PHP, CSS, JAVASCRIPT, etc}
\cvitem{Image Proc.}{Photoshop, Illustrator, ImageJ, Adobe Premiere, etc}
\cvitem{Data Ming}{Weka, Matlab, Numpy, etc}
\cvitem{3D Modeling}{Blender, 3DS Max, etc}
\cvitem{Presentation}{\LaTeX, OpenOffice, Word, PowerPoint, etc}

\section{荣誉}
\cvlistitem{Prize "EGIDE" 法国对外国留学生}


\section{发表}
%\renewcommand{\listitemsymbol}{-~}            % change the symbol for lists

% Publications from a BibTeX file without multibib
%  for numerical labels: \renewcommand{\bibliographyitemlabel}{\@biblabel{\arabic{enumiv}}}
%  to redefine the heading string ("Publications"): \renewcommand{\refname}{Articles}
\nocite{*}
\bibliographystyle{plain}
\bibliography{publications}                   % 'publications' is the name of a BibTeX file

\section{推荐人}
\cvlistitem{ Dr. Catherine Pelachaud  \url{http://perso.telecom-paristech.fr/~pelachau/}}
\cvlistitem{Dr. Tamy Boubekeur  \url{http://perso.telecom-paristech.fr/~boubek/}}
\cvlistitem{Dr. Elmar Eisemann  \url{http://graphics.tudelft.nl/~eisemann/}}
